\documentclass[12pt,a4paper]{article}
% Pacotes Essenciais
\usepackage[utf8]{inputenc}
\usepackage[brazil]{babel}
\usepackage[T1]{fontenc}
\usepackage{geometry}
\usepackage{setspace}
\usepackage{indentfirst}
\usepackage{booktabs}
\usepackage{array}

% Configurações ABNT (Margens e Espaçamento)
\geometry{
    a4paper,
    % ABNT NBR 14724: 3cm superior e esquerda; 2cm inferior e direita.
    left=3cm,
    right=2cm,
    top=3cm,
    bottom=2cm
}

\setlength{\parindent}{1.25cm} % Recuo de 1,25 cm para parágrafos (alinhado a NBR 14724)
\onehalfspacing % Espaçamento de 1,5 entre linhas

% Não é necessário adicionar comandos de fonte. O padrão (Computer Modern Roman) é substituído 
% por uma fonte serifada próxima ao Times New Roman quando usando a codificação T1.

% Definição de comando para o título de seção formatado (sem número, em negrito e caixa alta)
\newcommand{\secao}[1]{\section*{{\large\textbf{#1}}}} 

\begin{document}

\begin{center}
    \textbf{\Large CONTRATO SOCIAL}\\[0.5cm]
    \textbf{\large MORRISON KÜHLSEN CONSULTORIA E PESQUISA ESTATÍSTICA LTDA.}
\end{center}

\vspace{1cm}

\noindent\textbf{DANTE BERTUZZI}, brasileiro, [ESTADO CIVIL], [REGIME DE BENS, se casado], nascido em [DD/MM/AAAA], estatístico, inscrito no CPF sob o nº [XXX.XXX.XXX-XX], residente e domiciliado na [Logradouro completo], Bairro [Nome do Bairro], [Complemento], Petrolina -- PE, CEP [XX.XXX-XXX].

\vspace{0.5cm}

\noindent Resolve constituir uma sociedade limitada unipessoal, mediante as condições e cláusulas seguintes:

\vspace{0.5cm}

---
\secao{DO NOME EMPRESARIAL (ART. 1.158, § 2º, DO CÓDIGO CIVIL)}

\noindent\textbf{Cláusula Primeira} -- A sociedade adotará o seguinte nome empresarial: \textbf{MORRISON KÜHLSEN LTDA.}

---
\secao{DA SEDE (ART. 997, II, DO CÓDIGO CIVIL)}

\noindent\textbf{Cláusula Segunda} -- A sociedade terá sua sede no seguinte endereço: [Logradouro completo], [Número], [Bairro], Petrolina -- PE, CEP [XX.XXX-XXX].

---
\secao{DO OBJETO SOCIAL (ART. 997, II, E ART. 1.053, DO CÓDIGO CIVIL)}

\noindent\textbf{Cláusula Terceira} -- A sociedade terá por objeto o exercício das seguintes atividades econômicas:

\begin{itemize}
    \item \textbf{Atividade Principal:} Pesquisa e desenvolvimento experimental em ciências sociais e humanas (CNAE 72.20-3-00);
    
    \item \textbf{Atividades Secundárias:}
    \begin{itemize}
        \item Desenvolvimento de programas de computador sob encomenda (CNAE 62.01-5-01);
        \item Desenvolvimento e licenciamento de programas de computador customizáveis (CNAE 62.02-3-00);
        \item Tratamento de dados, provedores de serviços de aplicação e hospedagem na internet (CNAE 63.11-9-00);
        \item Outras atividades de ensino não especificadas anteriormente (CNAE 85.99-6-99);
        \item Serviços de estatística, análise de dados, consultoria estatística e pesquisas aplicadas.
    \end{itemize}
\end{itemize}

\noindent\textbf{Parágrafo único.} Em estabelecimento eleito como Sede (Matriz) serão exercidas todas as atividades constantes do objeto social.

---
\secao{DO INÍCIO DAS ATIVIDADES E DO PRAZO (ART. 997, IV, DO CÓDIGO CIVIL)}

\noindent\textbf{Cláusula Quarta} -- A sociedade iniciará suas atividades a partir de [DD/MM/AAAA] e seu prazo de duração é indeterminado.

---
\secao{DO CAPITAL SOCIAL (ART. 997, III, E ART. 1.055, DO CÓDIGO CIVIL)}

\noindent\textbf{Cláusula Quinta} -- O capital social é de R\$ 100.000,00 (cem mil reais), dividido em 100.000 (cem mil) quotas, no valor nominal de R\$ 1,00 (um real) cada uma, formado por R\$ 100.000,00 (cem mil reais) em moeda corrente do País.

\noindent\textbf{Parágrafo único.} O capital encontra-se subscrito e integralizado pelo sócio único da seguinte forma:

\vspace{0.3cm}

\begin{center}
\begin{tabular}{lccc}
\toprule
\textbf{SÓCIO} & \textbf{Nº de Quotas} & \textbf{Valor (R\$)} & \textbf{Percentual} \\
\midrule
DANTE BERTUZZI & 100.000 & 100.000,00 & 100\% \\
\midrule
\textbf{TOTAL} & \textbf{100.000} & \textbf{100.000,00} & \textbf{100\%} \\
\bottomrule
\end{tabular}
\end{center}

\vspace{0.3cm}

---
\secao{DA ADMINISTRAÇÃO (ART. 1.060 E ART. 1.061, DO CÓDIGO CIVIL)}

\noindent\textbf{Cláusula Sexta} -- A administração da sociedade será exercida pelo sócio \textbf{DANTE BERTUZZI}, que representará legalmente a sociedade e poderá praticar todo e qualquer ato de gestão pertinente ao objeto social.

\noindent\textbf{Parágrafo único.} Não constituindo o objeto social, a alienação ou a oneração de bens imóveis depende de autorização do sócio único.

---
\secao{DO BALANÇO PATRIMONIAL (ART. 1.078, DO CÓDIGO CIVIL)}

\noindent\textbf{Cláusula Sétima} -- Ao término de cada exercício, em \textbf{31 de dezembro}, o administrador prestará contas justificadas de sua administração, procedendo à elaboração do inventário, do balanço patrimonial e do balanço de resultado econômico, cabendo ao sócio os lucros ou perdas apuradas.

---
\secao{DA DECLARAÇÃO DE DESIMPEDIMENTO DE ADMINISTRADOR (ART. 1.011, § 1º, DO CÓDIGO CIVIL)}

\noindent\textbf{Cláusula Oitava} -- O administrador da empresa declara, sob as penas da lei, que não está impedido de exercer a administração da empresa, por lei especial, ou em virtude de condenação criminal, ou por se encontrar sob os efeitos dela, a pena que vede, ainda que temporariamente, o acesso a cargos públicos; ou por crime falimentar, de prevaricação, peita ou suborno, concussão, peculato, ou contra a economia popular, contra o sistema financeiro nacional, contra normas de defesa da concorrência, contra as relações de consumo, fé pública, ou a propriedade.

---
\secao{DO ENQUADRAMENTO (ART. 3º, DA LEI COMPLEMENTAR Nº 123/2006)}

\noindent\textbf{Cláusula Nona} -- O sócio declara que a sociedade se enquadra como \textbf{Microempresa -- ME}, nos termos da Lei Complementar nº 123, de 14 de dezembro de 2006, e que não se enquadra em qualquer das hipóteses de exclusão relacionadas no § 4º do art. 3º da mencionada lei.

---
\secao{DO PRO LABORE (ART. 28, DA LEI Nº 8.212/91 E IN SRF)}

\noindent\textbf{Cláusula Décima} -- O sócio poderá fixar uma retirada mensal, a título de \textbf{pro labore} para si, na condição de sócio administrador, observadas as disposições regulamentares pertinentes.

---
\secao{DO FORO (ART. 63, DO CÓDIGO DE PROCESSO CIVIL)}

\noindent\textbf{Cláusula Décima Primeira} -- A parte elege o foro da Comarca de \textbf{Petrolina, Estado de Pernambuco}, para dirimir quaisquer dúvidas decorrentes do presente instrumento contratual, bem como para o exercício e cumprimento dos direitos e obrigações resultantes deste contrato, renunciando a qualquer outro, por mais privilegiado que possa ser.

\vspace{1cm}

\noindent E, por estar assim constituída, assina o presente instrumento particular, em via única.

\vspace{1cm}

\noindent Petrolina -- PE, [DD] de [mês] de [AAAA].

\vspace{2cm}

\noindent\rule{8cm}{0.4pt}\\
\textbf{DANTE BERTUZZI}\\
Sócio Único e Administrador\\
CPF: [XXX.XXX.XXX-XX]

\vspace{2cm}

\noindent Visto: \rule{5cm}{0.4pt} (OAB/PE XXXX)

\end{document}